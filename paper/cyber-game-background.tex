% what are cyber games
An extensive literature now exists that analyzes cyber security using
game theory. However, there has been very little work analyzing realistic
games. Games can be categorized by various independent dimensions:
static or dynamic, complete or incomplete, perfect
or imperfect. Static games are games in which the order of the moves
do not matter; these are generally viewed as games where the players
move simultaneously. Non-static, or dynamic, games are more realistic
for modeling cyber conflict. In a game of complete information, each
player knows the payoff function of each of the other players, whereas
this is not the case for games of incomplete information. Finally, in
games of perfect information, each player observes (without error) all
actions taken and their outcomes. A notable feature of cyber conflict
is that each agent attempts to remain undetected; games where actions
or outcomes may be unobserved or observed with error or uncertainty
are called games of imperfect information. In this paper, we develop a
dynamic game of complete but imperfect information. For useful surveys
that describe the literature along this taxonomy,
see~\cite{liang2013game} and~\cite{roy2010survey}.

% *** In Liang, static game refs: 47, 46, 11, 19, 17, 45, 33, 18

% static games and repeated static games
A number of researchers have studied (or primarily focused on) static
games, such as~\cite{chen2009game,grossklags2009uncertainty,he2012game,liu2006bayesian}. These
games have the significant advantage that they can be analyzed (e.g.,
Nash equilibria can be computed) using existing methods. In some cases
(e.g.,~\cite{nguyen2009security}) the repeated game is studied, which
adds some realism, but fails to capture the essence of cyber attacks
where a sequence of small gains may lead to a dominant position.

% *** In Liang, dynamic, perfect: 14, 16, 39, 42, 48, 49

% dynamic games with perfect information
Among dynamic games, it is common to assume perfect information where
each player can observe the actions of the others. For example,
in~\cite{luo2010game, nguyen2009security, patcha2004game, 
  sagduyu2011jamming, shamma2005dynamic}, attacks are always
detected. This renders the results inapplicable to real cyber
conflicts, since cyber attacks are often undetected.

% *** In Liang, dynamic, imperfect: 8, 20, 21, 22, 26, 29,
% 30, 31, 36, 37, 44

% dynamic games with imperfect information (skip)
Almost all work involving dynamic games with imperfect information
focuses on so-called stochastic games. In these games
(e.g.,~\cite{alpcan2006intrusion, lye2005game, nguyen2009stochastic,
  sallhammar2006stochastic, xiaolin2008markov}), there is a finite
state space, and transition probabilities are determined by the
simultaneous move choices made by the players. Special methods have
been developed to find optimal strategies for these games, such as the
Q-Learning method~\cite{hu2003nash} and methods from competitive
Markov Decision Processes~\cite{filar2012competitive}. However, these
methods scale poorly with problem size, so we adopt different methods
in this work.

% key issues; we address them
The survey~\cite{roy2010survey} describes the lack of realism in cyber
games as a key issue in previous research.
\begin{quote}
  Some of the limitations of the present research are: (a) Current
  stochastic game models only consider perfect information and
  assume that the defender is always able to detect attacks; (b)
  Current stochastic game models assume that the state transition
  probabilities are fixed before the game starts \ldots; 
  (c) Current game models assume that the players' actions
  are synchronous, which is not always realistic; (d) Most models are
  not scalable with the size and complexity of the system under
  consideration.
\end{quote}
In this work, we address the first three of these concerns.

One issue all previous games have is that payoff functions must be
assigned to certain outcomes. These numbers typically must be chosen
to represent some intuitive notion of the benefit of each outcome, but
the payoffs, being very difficult to validate, must generally be
accepted as notional. In our game, we instead define a win condition. Namely,
the attacker who owns the most assets at the end of a given number of
time steps is the winner. This view of the payoff directly correlates
with the real-world goals of attackers and is therefore more
defensible.

Another key distinction is that {\em all} previous work we are aware
of analyzes an attacker-defender situation. We analyze a two-attacker
situation where two adversaries are attempting to control a common
pool of resources. This allows for a symmetry in the strategies and
avoids having to construct two different kinds of strategies (i.e.,
attackers and defenders). Consequently, the analysis and comparison of
strategies is simplified. Also, adding defenders at a future time will
be possible by augmenting the game with ownership and defensive moves
(e.g., clean-booting).

Furthermore, the complexity of our game makes it very difficult to use
the strategy optimization methods popular in the literature. Instead,
we adopt the strategy analysis methods used in games such as
chess. These include using an evaluation function and performing
$k$-ply searches, as described in~\cite{levy2009computer,shannon1950programming}.



