% what are cyber games
An extensive literature now exists that analyzes cyber security using
game theory. However, there is very little in terms of realistic
games. Games can be categorized by various independent dimensions:
static v. dynamic, complete v. incomplete, perfect
v. imperfect. Static games are games in which the order of the moves
do not matter; these are generally viewed as games where the players
move simultaneously. Non-static, or dynamic, games are more realistic
for modeling cyber conflict. In a game of complete information, each
player knows the payoff function of each of the other players, whereas
this is not the case for games of incomplete information, which may be
more realistic. Finally, in games of perfect information, each player
observes (without error) all actions taken and their outcomes. A
notable feature of cyber conflict is that each agent attempts to
remain undetected; games where actions or outcomes may be unobserved
or observed with error or uncertainty are called games of imperfect
information. In this paper, we develop a dynamic game of complete but
imperfect information. For useful surveys that describe the literature
along this taxonomy, see~\cite{liang2013game}
and~\cite{roy2010survey}. 

% static games and repeated static games
A number of researchers have studies static games~\cite{}. These games
have the significant advantage that they can be analyzed (e.g., Nash
equilibria can be computed) using existing methods. In some cases
(more \cites{}?) the repeated game is studied, which adds some realism,
but fails to capture the essence of cyber attacks where a sequence of
small gains leads to a dominant position.

% dynamic games with perfect information
Among dynamic games, it is common to assume complete and perfect
information where each player can observe the actions of the
others. For example, in~\cite{} attacks are always detected. This
renders the results inapplicable to real cyber conflicts, since cyber
attacks are designed to be (and often are) undetectable.

% dynamic games with imperfect information

% key issues; we address them
The survey~\cite{roy2010survey} describes the lack of realism in cyber
games.
\begin{quote}
  Some of the limitations of the present research are: (a) Current
  stochastic game models only consider perfect information and
  assume that the defender is always able to detect attacks; (b)
  Current stochastic game models assume that the state transition
  probabilities are fixed before the game starts and these
  probabilities can be computed from the domain knowledge and past
  statistics; (c) Current game models assume that the players’ actions
  are synchronous, which is not always realistic; (d) Most models are
  not scalable with the size and complexity of the system under
  consideration.
\end{quote}
In this work, we address all of these concerns.

One issue they seem to all have is that payoff functions must be
assigned to certain outcomes. These numbers typically must be chosen
to represent some intuitive notion of the benefit of each outcome, but
the choices, begin very difficult to validate, must generally be
accepted as notional. In our game, we define a win condition. Namely,
the attacker who owns the most assets at the end of a given number of
time steps is the winner. This view of the payoff directly correlates
with the real-world goals of attackers and is therefore more
defensible.

Another key distinction is that {\em all} previous work we are aware
of analyzes an attacker-defender situation. We analyze a two-attacker
situation where two adversaries are attempting to control a common
pool of resources.
