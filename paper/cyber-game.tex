% This is "sig-alternate.tex" V2.0 May 2012
% This file should be compiled with V2.5 of "sig-alternate.cls" May 2012
%
% This example file demonstrates the use of the 'sig-alternate.cls'
% V2.5 LaTeX2e document class file. It is for those submitting
% articles to ACM Conference Proceedings WHO DO NOT WISH TO
% STRICTLY ADHERE TO THE SIGS (PUBS-BOARD-ENDORSED) STYLE.
% The 'sig-alternate.cls' file will produce a similar-looking,
% albeit, 'tighter' paper resulting in, invariably, fewer pages.
%
% ----------------------------------------------------------------------------------------------------------------
% This .tex file (and associated .cls V2.5) produces:
%       1) The Permission Statement
%       2) The Conference (location) Info information
%       3) The Copyright Line with ACM data
%       4) NO page numbers
%
% as against the acm_proc_article-sp.cls file which
% DOES NOT produce 1) thru' 3) above.
%
% Using 'sig-alternate.cls' you have control, however, from within
% the source .tex file, over both the CopyrightYear
% (defaulted to 200X) and the ACM Copyright Data
% (defaulted to X-XXXXX-XX-X/XX/XX).
% e.g.
% \CopyrightYear{2007} will cause 2007 to appear in the copyright line.
% \crdata{0-12345-67-8/90/12} will cause 0-12345-67-8/90/12 to appear in the copyright line.
%
% ---------------------------------------------------------------------------------------------------------------
% This .tex source is an example which *does* use
% the .bib file (from which the .bbl file % is produced).
% REMEMBER HOWEVER: After having produced the .bbl file,
% and prior to final submission, you *NEED* to 'insert'
% your .bbl file into your source .tex file so as to provide
% ONE 'self-contained' source file.
%
% ================= IF YOU HAVE QUESTIONS =======================
% Questions regarding the SIGS styles, SIGS policies and
% procedures, Conferences etc. should be sent to
% Adrienne Griscti (griscti@acm.org)
%
% Technical questions _only_ to
% Gerald Murray (murray@hq.acm.org)
% ===============================================================
%
% For tracking purposes - this is V2.0 - May 2012

\documentclass{sig-alternate}

\begin{document}
%
% --- Author Metadata here ---
\conferenceinfo{CISR}{'15 Oak Ridge, Tennessee USA}
%\CopyrightYear{2007} % Allows default copyright year (20XX) to be over-ridden - IF NEED BE.
%\crdata{0-12345-67-8/90/01}  
% Allows default copyright data (0-89791-88-6/97/05) to be over-ridden - IF NEED BE.
% --- End of Author Metadata ---

\title{Game Theoretic Analysis Can Suggest Best Practices for Cyber
  Conflicts\titlenote{(Produces the permission block, and
copyright information). For use with
SIG-ALTERNATE.CLS. Supported by ACM.}}
% \subtitle{[Extended Abstract]
% \titlenote{A full version of this paper is available as
% \textit{Author's Guide to Preparing ACM SIG Proceedings Using
% \LaTeX$2_\epsilon$\ and BibTeX} at
% \texttt{www.acm.org/eaddress.htm}}}


%
% You need the command \numberofauthors to handle the 'placement
% and alignment' of the authors beneath the title.
%
% For aesthetic reasons, we recommend 'three authors at a time'
% i.e. three 'name/affiliation blocks' be placed beneath the title.
%
% NOTE: You are NOT restricted in how many 'rows' of
% "name/affiliations" may appear. We just ask that you restrict
% the number of 'columns' to three.
%
% Because of the available 'opening page real-estate'
% we ask you to refrain from putting more than six authors
% (two rows with three columns) beneath the article title.
% More than six makes the first-page appear very cluttered indeed.
%
% Use the \alignauthor commands to handle the names
% and affiliations for an 'aesthetic maximum' of six authors.
% Add names, affiliations, addresses for
% the seventh etc. author(s) as the argument for the
% \additionalauthors command.
% These 'additional authors' will be output/set for you
% without further effort on your part as the last section in
% the body of your article BEFORE References or any Appendices.

\numberofauthors{6} %  in this sample file, there are a *total*
% of EIGHT authors. SIX appear on the 'first-page' (for formatting
% reasons) and the remaining two appear in the \additionalauthors section.
%
\author{
% You can go ahead and credit any number of authors here,
% e.g. one 'row of three' or two rows (consisting of one row of three
% and a second row of one, two or three).
%
% The command \alignauthor (no curly braces needed) should
% precede each author name, affiliation/snail-mail address and
% e-mail address. Additionally, tag each line of
% affiliation/address with \affaddr, and tag the
% e-mail address with \email.
%
% 1st. author
\alignauthor
Erik M. Ferragut\\
       \affaddr{Oak Ridge National Laboratory}\\
       \affaddr{Oak Ridge, Tennessee}\\
       % \email{ferragutem@ornl.gov}
% 2nd. author
\alignauthor
Andrew Brady\\
       \affaddr{Jefferson Middle School}\\ 
       \affaddr{Oak Ridge, Tennessee}\\
% 3rd. author
\alignauthor
Ethan Brady\\
       \affaddr{Oak Ridge High School}\\ 
       \affaddr{Oak Ridge, Tennessee}\\
% 4th. author
\and
\alignauthor
Jacob Ferragut\\
       \affaddr{Oak Ridge High School}\\ 
       \affaddr{Oak Ridge, Tennessee}\\
% 5th. author
\alignauthor
Nathan Ferragut\\
       \affaddr{Oak Ridge High School}\\ 
       \affaddr{Oak Ridge, Tennessee}\\
% 6th. author
\alignauthor
Max Wildgruber\\
       \affaddr{Oak Ridge High School}\\ 
       \affaddr{Oak Ridge, Tennessee}\\
}

% There's nothing stopping you putting the seventh, eighth, etc.
% author on the opening page (as the 'third row') but we ask,
% for aesthetic reasons that you place these 'additional authors'
% in the \additional authors block, viz.
% \additionalauthors{Additional authors: John Smith (The Th{\o}rv{\"a}ld Group,
% email: {\texttt{jsmith@affiliation.org}}) and Julius P.~Kumquat
% (The Kumquat Consortium, email: {\texttt{jpkumquat@consortium.net}}).}
\date{16 October 2015}
% Just remember to make sure that the TOTAL number of authors
% is the number that will appear on the first page PLUS the
% number that will appear in the \additionalauthors section.

\maketitle
\begin{abstract}
Game theory + formalization of cyber conflict $\implies$ best
practices for cyber conflict
\end{abstract}

% % A category with the (minimum) three required fields
% \category{H.4}{Information Systems Applications}{Miscellaneous}
% %A category including the fourth, optional field follows...
% \category{D.2.8}{Software Engineering}{Metrics}[complexity measures, performance measures]

% \terms{Theory}

% \keywords{ACM proceedings, \LaTeX, text tagging}

\section{Introduction}
Computers connected to the Internet are potential targets for criminals, botnets, and national cyber capabilities. Cyber conflicts result from competing interests struggling to control the same assets. If one side of the conflict is manually assessing the situation, reasoning about the risks, making decisions, and executing attacks it will likely be overwhelmed if the other side adopts an automated approach in which algorithms are used to make every choice. The speed of human-in-the-loop decision making is likely to be several orders of magnitude slower than a well tuned algorithm. The future of cyber offense and defense undoubtedly depends on the development of advanced, scalable, and effective algorithms for decision-making in real-time. Moreover, the organization that first masters this technology will likely obtain a decisive advantage over their adversaries. 

A number of challenges remain before effective cyber conflict decision algorithms can be developed. Current methods lack a framework with which to automate reactions in realistic scenarios. In particular, what are relevant assets and capabilities? How can they be used? What is the relative benefit of each choice? And, given answers to these questions, how can the best actions be chosen and coordinated? These questions are the province of game theory.

In this paper, we take a game-theoretic approach to developing a prototype for intelligent, automated attack and defense. Previous work in this direction has analyzed simplified abstractions of cyber situations. We go further in this work by (1) creating a new game, HackAttack, that can be used to model larger, more complex conflicts than previous work, and (2) comparing and analyzing a number of strategies for HackAttack. These strategies use tree searches to maximize an evaluation function. We show that a simple evaluation function in a complex game can lead to interesting and nuanced strategies. This provides a promising path forward for developing a method for intelligent cyber conflict automation.

In Section~\ref{sec:background}, we review the extensive literature on applications of game theory to cyber security, and emphasize how this work extends it. In Section~\ref{sec:hackattack}, we introduce our new game, HackAttack, and describe its rules and concepts in detail. To play the game, the player strategies are created using tree searches and an evaluation function, which are described in Section~\ref{sec:strategies}. Strategies are analyzed in simulated games, as described in Section~\ref{sec:results}. In Section~\ref{sec:discussion} we summarize the lessons learned and implications for future work in regards to searching trees to select moves in cyber conflicts. We summarize our conclusions in Section~\ref{sec:conc}.

\section{Background}
Liang~\cite{liang2013game} is a survey.  It gives a summary of types of games broken down by cooperative v. non-cooperative, complete v. incomplete information (does each player know the other players' payoff functions), perfect v. imperfect information (do players know all previous moves and outcomes?), static v. dynamic game (one simultaneous move or consecutive moves). By this breakdown, our game is non-cooperative, has complete but imperfect information, and is a dynamic game.

Sallhammar~\cite{sallhammar2007using} is Karin Sallhammar's thesis that has one introductory paragraph and six chapters, each one a paper. The title given is the first paper (Paper A).  Paper C is~\cite{sallhammar2005incorporating}. Paper D is~\cite{sallhammar2006stochastic} (which is probably essentially the same as~\cite{sallhammar2006towards}). Paper F is~\cite{sallhammar2007framework}. (Papers B and E are about attackers' expected behavior and real-time risk assessment.)




And so on...


Roy~\cite{roy2010survey}.

Tambe~\cite{tambe2011security}

Lye~\cite{lye2005game}




Bier~\cite{bier2006game}

Luo~\cite{luo2010game}

Fielder~\cite{fielder2014game}

Backhaus~\cite{backhaus2013cyber}

Zonouz~\cite{zonouz2014rre}

Dunlavy~\cite{dunlavy2009mathematical}

Grossklags~\cite{grossklags2009uncertainty}

Shiva~\cite{shiva2010stochastic}

Carroll~\cite{carroll2011game}

Kiekintveld~\cite{kiekintveld2009computing}

Manshaei~\cite{manshaei2013game}

Alpcan~\cite{alpcan2006intrusion}

Kiekintveld~\cite{kiekintveld2011approximation}

Bensoussan~\cite{bensoussan2010game}

Sallhammar~\cite{sallhammar2007framework}

Sallhammar~\cite{sallhammar2005incorporating}

Farhang~\cite{farhang2014dynamic}

Van~\cite{van2013flipit}

Shen~\cite{shen2011survey}

Letchford~\cite{letchford2011computing}

Wu~\cite{wu2010modeling}

Chan~\cite{chan2012interdependent}

Bowers~\cite{bowers2012defending}

Gueye~\cite{gueye2012towards}

Sahinoglu~\cite{sahinoglu2012game}

Shiva~\cite{shiva2012holistic}

He~\cite{he2012game}

Schlicher~\cite{schlicher2012information}

Beard~\cite{beard2012using}

Simmons~\cite{simmons2013adapt}

Leslie~\cite{leslie2015threshold}

Abbasvand~\cite{abbasvand2013survey}

Jones~\cite{jones2013asymmetric}


\section{Definition of HackAttack}
We developed the game, HackAttack, specifically for this research to simplify cyber conflict into an analyzable game while avoiding oversimplification. 
In HackAttack, players battle over a pool of resources, similar to how two actors might attempt to grow their botnets. 
The winner is the player who has the most computers after a set time frame. 
The game is turn based in a fixed player order, and players take one action each turn for each resource they control. The resources are a collection of ``neutral'' computers accessible from the Internet. 

The rules of this game apply to any number of players, and the game has been played with up to five players to test the game mechanics. However, for the purposes of this analysis, only two players were used. 

{\em Computers and Accounts.}
The game is played with $5p$ computers where $p$ is the number of players.
For example, in a two-player game, there are 10 available computers.
Each player has a starting computer on which they have one account.
Accounts represent the presence a player has on a computer. Each player begins with one account on their starting computer
More accounts make it easier to remove the other player off a computer and to resist being removed off a computer.
We say that a player {\em controls} a computer if they have at least one account on it.
The most accounts any player can have on any machine is four. 
Multiple players can have a presence on a computer, with each of them having up to four accounts on it at the same time. 

{\em Exploits.}
Exploits are named by the operating system (OS) they target and their power level.  At the start of the game, computers are randomly, but equiprobably designated a certain OS, indicated by four types, 0 through 3. (These can be thought to stand for Linux, Mac, Solaris, and Windows, for example, but the specific labels have no bearing on the game.) 

The power levels for an exploit range from 0 to 14. The chance of getting an exploit with power level $n$ is $2^{-(n+1)}$. Therefore an exploit with power 0 has a 50\% chance of being found, an exploit with power 1 a 25\% chance, all the way to exploits with power 14, which has about a 0.003\% chance of being found\footnote{Technically, the probabilities are then normalized to sum to 1 by dividing by (100 - 0.003)\%.}. Each player starts out with four exploits, and there is a 1/6 chance of gaining a random exploit each round. 

{\em Vulnerabilities and Patches.}
Computers at the start of the game are patched against three exploits, unknown to the players, representing the antivirus software the host user already has. These patches are chosen with the same probability as the exploits for players; weaker ones are more common. 


{\em Actions.}
On their turn, each player gets to assign one action to each computer they control.
Actions are types of attacking, defending, and scouting. These moves are summarized in Table~\ref{tab:moves}.
	
\begin{table*}
\begin{center}
\begin{tabular}{|l|l|r|r|}
\hline 
\highlightrow \multicolumn{4}{|c|}{Attack} \\
\hline 
{\tt Hacking} & Use an exploit from one computer to gain an account on another &  20\% P.D. & Remote \\
\hline 
\highlightrow \multicolumn{4}{|c|}{Defense} \\
\hline 
{\tt Backdooring} & Add more ways to control a computer & 15\% P.D.  & Local \\
{\tt Cleaning} & Remove other player's accounts from a computer & 100\% P.D. & Local \\
{\tt Patching} & Block all future uses of an exploit on a computer & 25\% P.D.  & Local \\
\hline 
\highlightrow \multicolumn{4}{|c|}{Scouting} \\
\hline 
{\tt Reconning} & Identify a computer's OS and vulnerabilities & 5\% P.D. & Remote \\
{\tt Scanning} & Use an account on a computer to find the presence of other players & 30\% P.D. 
& Local \\
\hline
\end{tabular}
\caption{Summary of available moves in HackAttack. PD = Probability of Detection. {\tt Hacking} and {\tt recon} affect a targeted computer from a controlled computer. The other actions affect the controlled computer itself.\label{tab:moves}}
\end{center}
\end{table*}

Attacking, in the game called {\tt hacking}, is using an exploit to gain one account on a computer.
An exploit for a specific OS only works for that one OS.  
Exploits are the only form of attack, and using different exploits is the only way to change how you attack. If the {\tt hacking} succeeds, the one carrying out the hack gains an account on the target machine.  
A successful hack puts one account on the computer targeted, up to the maximum number of accounts. 
If the {\tt hacking} fails, the attacker learns no additional information about the target. 

Defense is accomplished by {\tt patching} to decrease the likelihood of an opponent successfully attacking your computer, {\tt cleaning} to remove your opponent from a computer, or {\tt backdooring} to making it harder for you to be removed off a computer. 
{\tt Backdooring} puts one more account on a computer that you own. 
{\tt Cleaning} removes enemy accounts off a computer equal to the accounts you have on that computer.  Removing a player from a computer is important because only computers with at least one player account on it will count toward a player's final computer total. After {\tt cleaning}, you know how many enemy accounts were removed.  The different outcomes of this have varying meanings, and certain results make it unclear if the opponent is still on the computer. {\tt Cleaning} is especially useful after you detect an opponent using that machine.
 {\tt Patching} makes a computer you own permanently safe against an exploit you choose. Since you only know about the exploits you own and use to hack, you can only patch exploits you own. 
One downside of {\tt patching} is that if you patch an exploit and are later removed from the computer, then you cannot use the exploit you patched to get back on that computer. 

The scouting functions, {\tt scanning} and {\tt reconning}, reveal information about a computer. {\tt Scanning} tells you who else, if anyone, is on the targeted computer you own and the amount of accounts they have on it.  It may be used before {\tt cleaning} to know if your {\tt clean} will entirely remove your opponent. Unlike {\tt cleaning}, {\tt scanning} tells you definitively if someone else is on a computer and if they will remain after a {\tt clean}.  {\tt Reconning} one computer from a controlled computer tells you the OS of the targeted computer and what exploits you have, if any, that can {\tt hack} it, but it does not indicate whether it is already occupied. This action is used to set up a {\tt hack} on the following turn because it can almost guarantee the {\tt hack}'s success.  If you are {\tt cleaned} off that computer and want to get back on, you would not need to {\tt recon} again because you would remember the computer's OS.

{\em Detection.}
It is important to note that each player only observes another player's action if it is ``detected''. Otherwise, they are aware only of their own actions and the results of those actions. Every action, therefore, has a detection probability based on how much it interacts with the computer.  {\tt Patching} has a 25\% detection rate, {\tt hacking} 20\%, {\tt backdooring} 15\%, {\tt reconning} 5\%, {\tt scanning} 30\%, and {\tt cleaning} 100\%. One important thing to note is that you can be detected by a player on the targeted machine, and, in the case of {\tt hacking} and {\tt reconning}, on the acting computer. 


{\em Win Condition.}
After 20 rounds, the player controlling the greatest number of computers is the winner, or a tie is declared if each player controls the same number of machines. A game is stopped early if one player is {\tt cleaned} off of every computer. 
Having at least one account on a computer makes it contribute to your total number of machines controlled at the end of the twenty rounds, and both players can have the same computer count toward their totals.  



\section{Evaluation Functions}
\input{cyber-game-evalfunctions.tex}
Include subsections?

\section{Results}

\input{../results-from-rota/table.tex}


\section{Discussion}
Our experiments produced both expected and unexpected 
outcomes. It was expected, and the experiments showed, that deeper
tree searches yield better performance. In particular, the Random
strategy, effectively a 0-ply search, performs the worst, as
expected. Still, it manages to defeat the other strategies
occasionally when it starts the game with a powerful
exploit. Furthermore, the 1- and 2-ply searches outperformed the 0-ply
search.  Unfortunately, due to the large breadth of possible moves,
searches of depth three or more were intractable with the available
computational resources; this is discussed below.

Other results were unexpected. Not only did the same evaluation
functions lead to different strategies, but each individual strategy
produced a nuanced variation in tactics depending on the power and
variety of exploits it
begins with. 
In general, more powerful exploits lead to more agressive
play (more attacks), and weaker exploits lead to more defensive play
(more securing of controlled computers). This shows that the
complexity of realistic cyber conflict may not require
complex algorithms. Instead, a simple evaluation function and a
$k$-ply search may suffice to produce highly effective, robust, and
nuanced strategies.

Another unexpected result was the clear dominance of the No-Response
strategy over the Random-Response strategy. This shows that, in this
example of extreme uncertainty about the opponent's situation, it is
more effective to consider one's own moves only. Attempting to average
or search over a wide range of opponent moves when each move has only
a very small probability only seems to help avoid ``knock-out''
moves. (For example, the Random-Response strategy was more likely than
the others to {\tt clean} a newly acquired computer, apparently in
order to avoid being {\tt cleaned} off by the other opponent should
they have happened to be on there first. This observation adds
credibility to the analysis of risk from unilateral actions common in the
literature, such as in attack trees~\cite{schneier1999attack}, attack
graphs~\cite{sheyner2002automated}, and attack Petri
nets~\cite{zakrzewska2011modeling}.

The main shortcoming of this approach is the fact that the
exploration of moves in a $k$-ply search tree is very slow and scales
exponentially with $k$. This made it impossible to explore the
implications of deeper searches. However, a number of methods
for accelerating and deepening the search are available. For example,
in early rounds of the game, there is typically a symmetry among unexplored
computers in the sense that what is known about each of them is the
same. As a result, they lead to the same scores being computed for
moves on each of those computers. 
Rather
than simply compute all of those scores in the tree, the symmetry in
the knowledge can be exploited to drastically reduce the exploration
required. Another approach to speed up the search involves being
more judicious on which branches of the tree to explore. For example,
if a first possible move is especially poor, the tree can be cut at
that point, eliminating all of the computations that belong to its
descendents. These and other heuristics for accelerating computation
have been well studied in the literature~\cite{levy2009computer}. They
have been applied to chess games to provide substantial speed-ups and
to enable much deeper searches since Claude Shannon's seminal work in
1950~\cite{shannon1950programming}.



\section{Conclusions}

We have shown that an application of game theoretic methods can be used
to analyze a more realistic cyber game than has been heretofore been
studied. In the process, we created strategies that exploit searches
of trees comprised of possible moves. Using simple functions that
evaluate the value of any game situation together with a tree
searching capability, we showed that the resulting strategies can
produce intelligent, nuanced strategies. For example, the resulting
strategies took into account the player's relative advantage based on
the quality of their exploits.  Also, we found that in cases of
extreme uncertainty, it is often better to ignore one's opponent's
possible moves.

One area of interest worthy of further analysis is the investigation
into the value of certainty. 
By creating an alternative
evaluation function that adds some weight to the information content in
the knowledge of the game, the strategy could emphasize an intrinsic
value of learning the OS of computers and the players on each
machine. Such an evaluation function might allow a 1-ply search to
mimic the strategies that were found to be so effective in the 2-ply
No-Response strategy.



%ACKNOWLEDGMENTS are optional
\section{Acknowledgments}
This section is optional; it is a location for you
to acknowledge grants, funding, editing assistance and
what have you.  In the present case, for example, the
authors would like to thank Gerald Murray of ACM for
his help in codifying this \textit{Author's Guide}
and the \textbf{.cls} and \textbf{.tex} files that it describes.

%
% The following two commands are all you need in the
% initial runs of your .tex file to
% produce the bibliography for the citations in your paper.
\bibliographystyle{abbrv}
\bibliography{game}  % sigproc.bib is the name of the Bibliography in this case
% You must have a proper ".bib" file
%  and remember to run:
% latex bibtex latex latex
% to resolve all references
%
% ACM needs 'a single self-contained file'!
%
%APPENDICES are optional

%\balancecolumns

\end{document}
