Computers connected to the Internet are potential targets for criminals, botnets, and national cyber capabilities. Cyber conflicts result from competing interests struggling to control the same assets. If one side of the conflict is manually assessing the situation, reasoning about the risks, making decisions, and executing attacks it will likely be overwhelmed if the other side adopts an automated approach in which algorithms are used to make every choice. The speed of human-in-the-loop decision making is likely to be several orders of magnitude slower than a well tuned algorithm. The future of cyber offense and defense undoubtedly depends on the development of advanced, scalable, and effective algorithms for decision-making in real-time. Moreover, the organization that first masters this technology will likely obtain a decisive advantage over their adversaries. 

A number of challenges remain before effective cyber conflict decision algorithms can be developed. Current methods lack a framework with which to automate reactions in realistic scenarios. In particular, what are relevant assets and capabilities? How can they be used? What is the relative benefit of each choice? And, given answers to these questions, how can the best actions be chosen and coordinated? These questions are the province of game theory.

In this paper, we take a game-theoretic approach to developing a prototype for intelligent, automated attack and defense. Previous work in this direction has analyzed simplified abstractions of cyber situations. We go further in this work by (1) creating a new game, HackAttack, that can be used to model larger, more complex conflicts than previous work, and (2) comparing and analyzing a number of strategies for HackAttack. These strategies use tree searches to maximize an evaluation function. We show that a simple evaluation function in a complex game can lead to interesting and nuanced strategies. This provides a promising path forward for developing a method for intelligent cyber conflict automation.

In Section~\ref{sec:background}, we review the extensive literature on applications of game theory to cyber security, and emphasize how this work extends it. In Section~\ref{sec:hackattack}, we introduce our new game, HackAttack, and describe its rules and concepts in detail. To play the game, the player strategies are created using tree searches and an evaluation function, which are described in Section~\ref{sec:strategies}. Strategies are analyzed in simulated games, as described in Section~\ref{sec:results}. In Section~\ref{sec:discussion} we summarize the lessons learned and implications for future work in regards to searching trees to select moves in cyber conflicts. We summarize our conclusions in Section~\ref{sec:conc}.