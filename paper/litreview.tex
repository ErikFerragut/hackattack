% This is "sig-alternate.tex" V2.0 May 2012
% This file should be compiled with V2.5 of "sig-alternate.cls" May 2012
%
% This example file demonstrates the use of the 'sig-alternate.cls'
% V2.5 LaTeX2e document class file. It is for those submitting
% articles to ACM Conference Proceedings WHO DO NOT WISH TO
% STRICTLY ADHERE TO THE SIGS (PUBS-BOARD-ENDORSED) STYLE.
% The 'sig-alternate.cls' file will produce a similar-looking,
% albeit, 'tighter' paper resulting in, invariably, fewer pages.
%
% ----------------------------------------------------------------------------------------------------------------
% This .tex file (and associated .cls V2.5) produces:
%       1) The Permission Statement
%       2) The Conference (location) Info information
%       3) The Copyright Line with ACM data
%       4) NO page numbers
%
% as against the acm_proc_article-sp.cls file which
% DOES NOT produce 1) thru' 3) above.
%
% Using 'sig-alternate.cls' you have control, however, from within
% the source .tex file, over both the CopyrightYear
% (defaulted to 200X) and the ACM Copyright Data
% (defaulted to X-XXXXX-XX-X/XX/XX).
% e.g.
% \CopyrightYear{2007} will cause 2007 to appear in the copyright line.
% \crdata{0-12345-67-8/90/12} will cause 0-12345-67-8/90/12 to appear in the copyright line.
%
% ---------------------------------------------------------------------------------------------------------------
% This .tex source is an example which *does* use
% the .bib file (from which the .bbl file % is produced).
% REMEMBER HOWEVER: After having produced the .bbl file,
% and prior to final submission, you *NEED* to 'insert'
% your .bbl file into your source .tex file so as to provide
% ONE 'self-contained' source file.
%
% ================= IF YOU HAVE QUESTIONS =======================
% Questions regarding the SIGS styles, SIGS policies and
% procedures, Conferences etc. should be sent to
% Adrienne Griscti (griscti@acm.org)
%
% Technical questions _only_ to
% Gerald Murray (murray@hq.acm.org)
% ===============================================================
%
% For tracking purposes - this is V2.0 - May 2012

\documentclass{sig-alternate}

\begin{document}
%
% --- Author Metadata here ---
\conferenceinfo{CISR}{'15 Oak Ridge, Tennessee USA}
%\CopyrightYear{2007} % Allows default copyright year (20XX) to be over-ridden - IF NEED BE.
%\crdata{0-12345-67-8/90/01}  
% Allows default copyright data (0-89791-88-6/97/05) to be over-ridden - IF NEED BE.
% --- End of Author Metadata ---

\title{Literature Review of Cyber Games}

% \subtitle{[Extended Abstract]
% \titlenote{A full version of this paper is available as
% \textit{Author's Guide to Preparing ACM SIG Proceedings Using
% \LaTeX$2_\epsilon$\ and BibTeX} at
% \texttt{www.acm.org/eaddress.htm}}}


%
% You need the command \numberofauthors to handle the 'placement
% and alignment' of the authors beneath the title.
%
% For aesthetic reasons, we recommend 'three authors at a time'
% i.e. three 'name/affiliation blocks' be placed beneath the title.
%
% NOTE: You are NOT restricted in how many 'rows' of
% "name/affiliations" may appear. We just ask that you restrict
% the number of 'columns' to three.
%
% Because of the available 'opening page real-estate'
% we ask you to refrain from putting more than six authors
% (two rows with three columns) beneath the article title.
% More than six makes the first-page appear very cluttered indeed.
%
% Use the \alignauthor commands to handle the names
% and affiliations for an 'aesthetic maximum' of six authors.
% Add names, affiliations, addresses for
% the seventh etc. author(s) as the argument for the
% \additionalauthors command.
% These 'additional authors' will be output/set for you
% without further effort on your part as the last section in
% the body of your article BEFORE References or any Appendices.

\numberofauthors{1} %  in this sample file, there are a *total*
% of EIGHT authors. SIX appear on the 'first-page' (for formatting
% reasons) and the remaining two appear in the \additionalauthors section.
%
\author{
% You can go ahead and credit any number of authors here,
% e.g. one 'row of three' or two rows (consisting of one row of three
% and a second row of one, two or three).
%
% The command \alignauthor (no curly braces needed) should
% precede each author name, affiliation/snail-mail address and
% e-mail address. Additionally, tag each line of
% affiliation/address with \affaddr, and tag the
% e-mail address with \email.
%
% 1st. author
\alignauthor
Erik M. Ferragut\\
       \affaddr{Oak Ridge National Laboratory}\\
       \affaddr{Oak Ridge, Tennessee}\\
       % \email{ferragutem@ornl.gov}
}

% There's nothing stopping you putting the seventh, eighth, etc.
% author on the opening page (as the 'third row') but we ask,
% for aesthetic reasons that you place these 'additional authors'
% in the \additional authors block, viz.
% \additionalauthors{Additional authors: John Smith (The Th{\o}rv{\"a}ld Group,
% email: {\texttt{jsmith@affiliation.org}}) and Julius P.~Kumquat
% (The Kumquat Consortium, email: {\texttt{jpkumquat@consortium.net}}).}
\date{16 October 2015}
% Just remember to make sure that the TOTAL number of authors
% is the number that will appear on the first page PLUS the
% number that will appear in the \additionalauthors section.

\maketitle
% \begin{abstract}
% Lit review for cyber conflict
% \end{abstract}

% % A category with the (minimum) three required fields
% \category{H.4}{Information Systems Applications}{Miscellaneous}
% %A category including the fourth, optional field follows...
% \category{D.2.8}{Software Engineering}{Metrics}[complexity measures, performance measures]

% \terms{Theory}

% \keywords{ACM proceedings, \LaTeX, text tagging}

\section{Detailed summary}

Liang~\cite{liang2013game} is a survey.  It gives a summary of types
of games broken down by cooperative v. non-cooperative, complete
v. incomplete information (does each player know the other players'
payoff functions), perfect v. imperfect information (do players know
all previous moves and outcomes?), static v. dynamic game (one
simultaneous move or consecutive moves). By this breakdown, our game
is non-cooperative, has complete but imperfect information, and is a
dynamic game. 


Sallhammar~\cite{sallhammar2007using} is Karin Sallhammar's thesis
that has one introductory paragraph and six chapters, each one a
paper. The title given is the first paper (Paper A).  Paper C
is~\cite{sallhammar2005incorporating}. Paper D
is~\cite{sallhammar2006stochastic} (which is probably essentially the
same as~\cite{sallhammar2006towards}). Paper F
is~\cite{sallhammar2007framework}. (Papers B and E are about
attackers' expected behavior and real-time risk assessment.)
Algorithms for solving stochastic games in [Som04].  In {\bf Paper A} the
game has a few states from secure to breached.  Attacker may increment
the state (toward breached) and defender may reset it to secure (0).
They model it as continuous time with exponential wait times driven by
selected parameters for the attacker and defender speeds and also
probabilities of success.  Treated like a game (using min-max) but the
defender has no choice, right?  Unclear.  Turns into a markov chain
problem where the limiting probabilities can be computed from the
transition matrix.  Becomes unclear and focuses on mean time to first
system breach (MTFSB).  Somehow, best attack rate is 0.25, which is
because the attacker prefers to not attack {\em when they would have
  been detected} over not attacking when they would not have been
detected.  Apparently requires knowing what would have happened, which
is very unrealistic.  But without that, they would be unable to
distinguish attack cases and would either always or never attack, I
think.  Paper B is like Paper A except it posits a set of three
resources and stochastically models attacks on them.

Bensoussan~\cite{bensoussan2010game} uses a {\em differential game}
similar to a SIS epidemiological model where no longer infected
machines can later again become infected. This system is then solved
by setting the right derivatives to zero (and using dynamic
programming). Not especially relevant despite being about a botnet
attacker and a network defender. Also, it seems no stochastics are
included.

Sallhammar~\cite{sallhammar2007framework}, also known as Paper F from
her thesis, is the culmination of several papers from 2002--2007. The
general framework appears to be based on using a Hidden Markov Model
to capture the gap between observations and true state as the system
evolves over time.

Lye~\cite{lye2005game} uses a stochastic game with two players
(attacker and defender), which works as follows.  There are a finite
collection of states.  At each time, both players choose a move, and
then they receive a reward that's a function of the state and the
chosen moves (different function for each player). Then the state is
updated according to transition probabilities depending on previous
state and player moves. Nash equilibria are computed (18 in all).
They point out the common complaint ``why not put all security on''
and say it is to preserve usability (at end of Section 6).

Roy~\cite{roy2010survey} is a survey that comes from Dasgupta's
lab. Their taxonomy has cooperative and non-cooperative games, with
the latter broken up as static or dynamic games. They further break up
the static and dynamic games, with dynamic ones being divided into
complete and not complete information and also perfect and imperfect
information. In the category most like HackAttack, complete and
imperfect information, they list just two references
(\cite{alpcan2006intrusion} and \cite{nguyen2009security}).  In the
closely related category of incomplete and imperfect information, two
more (\cite{alpcan2004game} and \cite{you2003kind}) are given. Their
discussion in Section 4.4 lists a number of issues that HackAttack
seems to address:

\begin{quotation}
Many of the current game-theoretic security approaches are based on
either static game models [23, 24] or games with perfect information
[25, 39, 4, 6] or games with complete information [31]. However, in
reality a network administrator often faces a dynamic game with
incomplete and imperfect information against the attacker. Some of the
current models involving dynamic game with incomplete and imperfect
information are specific to wireless networks [34] while a few others
[2, 40] do not consider a realistic attack scenario.

In particular, some of the limitations of the present research are:
(a) Current stochastic game models [25] only consider perfect
information and assume that the defender is always able to detect
attacks; (b) Current stochastic game models [25] assume that the state
transition probabilities are fixed before the game starts and these
probabilities can be computed from the domain knowledge and past
statistics; (c) Current game models assume that the players’ actions
are synchronous, which is not always realistic; (d) Most models are
not scalable with the size and complexity of the system under
consideration.
\end{quotation}

Alpcan~\cite{alpcan2006intrusion} is one of only a few examples to
consider dynamic games with imperfect information. They use three
players: attacker, defender, and purely stochastic sensors. They get a
zero-sum (for ease of analysis) 2-player Markov stochastic game. They
consider various levels of information, including defective
sensors. The attacker's options are attack and don't attack, which is
either detected or not detected, and the defender can either respond
or not respond. This is repeated in consecutive discrete
times. Specialized algorithms from the literature are used to
``solve'' the game. (The co-author T.~Ba{\c{s}}ar has a full book
called {\em Dynamic Noncooperative Game Theory}.)

Alpcan~\cite{alpcan2004game} considers both a discrete/finite version
and a continuous version. They haven't quite stumbled onto Markov
stochastic games, but they are building toward it. Basically an early
version of~\cite{alpcan2006intrusion}, in my opinion. They consider a
dynamic game, which is stochastic via a ``nature'' player (whom they
call fictitious play), and they attempt to solve it using simulations
and such.

Nguyen~\cite{nguyen2009security} considers repeated static games,
sets up a matrix game and solves it. As payoff matrices seem
arbitrary, the results are notional only.

Bier~\cite{bier2006game} is essentially a concept paper arguing for
combining reliability methods and game theory to study conflicts
related to terrorism (not necessarily cyber attacks). The interesting
idea is how some defenses (e.g., conspicuous alarm system) increase
risk for other defenders while other defenses (e.g., vaccines)
decrease risk for others. This suggests a dynamic that we did not
(yet) study directly.

In~\cite{luo2010game} a tree approach is used to handle a dynamic game
with essentially complete and perfect information, although they seem
to claim otherwise. They compare a greedy, a single step look-ahead,
and (probably) a further look ahead and show the last is best.

In~\cite{fielder2014game}, they find Nash equilibria and mixed Nash
equilibria for games and show they correspond to minimax
solutions. Also, they use reports of hacking incidents as a basis for
their games. There exist $n$ resources at baseline defense, and the
defender may harden $m < n$ of them. Each asset has a payoff for each
of attacker and defender for each case of being compromised or
not. They find Nash equilibrium. They prove that the defender's Nash
equilibrium is also their minimax strategy for that game (their
Theorem 1). (They also do a fast approximation of it using SVD.) They
have an example where the obvious solution is not as good as the game
theory solution.

In~\cite{grossklags2009uncertainty}, the authors focus on  quantifying
the value of information for rational and intelligent defenders. They
also compare sophisticated and unsophisticated defenders (as modeled
by different ways to find strategies).

\section{Categorization}

\begin{description}
\item[Surveys] \cite{liang2013game,roy2010survey}

\item[Concept Papers] Relationship to reliability~\cite{bier2006game}.

\item[Non-repeated Static Games] TBD

\item[Repeated Static Games] Nguyen et
  al.~\cite{nguyen2009security} consider repeated static matrix games
  (with apparently arbitrary matrices).


\item[Dynamic Games -- Differential] SIS epidemiological modeling
  by~\cite{bensoussan2010game}.


\item[Dynamic Games -- Stochastic] The work of Sallhammar et
  al.~\cite{sallhammar2007using, sallhammar2005incorporating,
    sallhammar2006stochastic, sallhammar2006towards,
    sallhammar2007framework} is based on stochastic
  games. Also,~\cite{lye2005game} where 18 Nash equilibria are
  computed. Several works by Ba{\c{s}}ar consider dynamic games with
  imperfect information~\cite{alpcan2006intrusion} in terms of a
  2-player zero-sum Markov stochastic game.

\item[Dynamic Games -- Other] Tree-based approach~\cite{luo2010game}. 

\end{description}

%
% The following two commands are all you need in the
% initial runs of your .tex file to
% produce the bibliography for the citations in your paper.
\bibliographystyle{abbrv}
\bibliography{game}  % sigproc.bib is the name of the Bibliography in this case
% You must have a proper ".bib" file
%  and remember to run:
% latex bibtex latex latex
% to resolve all references
%
% ACM needs 'a single self-contained file'!
%
%APPENDICES are optional

%\balancecolumns

\end{document}
